\setcounter{equation}{0}
\section{Первые интегралы}
\subsection{Первые интегралы автономных систем}
Пусть $\Omega \subset \mathbb R^{n}$ открыто и $f \in C^1(\Omega, \mathbb R^n)$. Снова рассматриваем автономную систему
\begin{equation}
    x' = f(x).
\end{equation}

\textbf{Определение.} Если $D \subset \Omega$ открыто, то \textit{первым интегралом} системы (1) называется непрерывно дифференцируемая функция $u\colon D \to \mathbb R$ такая, что $u(x(t)) \equiv$ const для любого такого решения $x\colon I \to \mathbb{R}$ системы (1), что $x(I) \subset D$.

\textbf{Замечание.} Первый интеграл всегда существует, например, $u \equiv$ const.

\textbf{Утверждение.} (критерий первого интеграла) Пусть $D \subset \Omega$ открыто и функция $u\colon D \to \mathbb R$ непрерывно дифференцируема. Тогда $u(\cdot)$ --- первый интеграл системы (1) $\Leftrightarrow$ $\left.\frac{du}{dt}\right|_{(1)}(x) \equiv 0$.

\textbf{Доказательство.} $\Rightarrow$: Берём $x_0 \in D$. Тогда существует решение $x\colon I \to \mathbb{R}^n$ задачи Коши
\[
    \begin{cases}
        x' = f(x) \\
        x(t_0) = x_0
    \end{cases}
\]
для некоторого $t_0 \in I$. При этом при необходимости можно сузить область определения $I$ так, чтобы $x(I) \subset D$.
Так как функция $u(x(t)) \equiv$ const, то по свойству производной в силу системы
\[
    \left.\frac{du}{dt}\right|_{(1)}(x(t)) \equiv \frac{du}{dt}(x(t)) \equiv 0.
\]
Подставляем $t = t_0$, получаем: $\left.\frac{du}{dt}\right|_{(1)}(x_0) = 0$. В силу произвольности выбора $x_0$ получили требуемое.

$\Leftarrow$: Берём произвольное решение $x\colon I \to \mathbb{R}$ системы (1) такое, что $x(I) \subset D$. Тогда
\[
\frac{du}{dt}(x(t)) \equiv \left.\frac{du}{dt}\right|_{(1)}(x(t)) \equiv 0,
\]
отсюда получаем, что $u(x(t)) \equiv$ const $\Rightarrow$ по определению $u(\cdot)$ --- первый интеграл системы (1).

\QED

\textbf{Примеры.}
\begin{itemize}
    \item Пусть $\Omega = \mathbb{R}^2$, $D = \mathbb{R}^2$. Рассмотрим систему $x' = x$. Её решение --- это $x_1=c_1e^t$, $x_2=c_2e^t$.
    На фазовом портрете траекториями будут все открытые лучи, выходящие из начала координат. Пусть $u(\cdot)$ --- первый интеграл. Возьмём конкретный луч, на нём $u \equiv c$ для какой-то константы $c$.
    По непрерывности $u(0) \equiv c$. Но тогда на всех лучах $u \equiv c$, значит, $u \equiv c$ на всём $D$ и других первых интегралов быть не может.
    \item Рассмотрим ту же самую систему, только теперь $D = \{(x_1, x_2)^T : x_1 > 0\}$. Покажем, что функция $u(x) = \frac{x_2}{x_1}$ является первым интегралом. Действительно,
    \[
        u(x(t)) \equiv \frac{c_2e^t}{c_1e^t} \equiv \frac{c_2}{c_1}.
    \]
    \item Рассмотрим систему
    \[
        \begin{cases}
            x_1' = -x_2 \\
            x_2' = x_1
        \end{cases}.
    \]
    Её решение выглядит так:
    \[
        \begin{cases}
            x_1 = r\cos(\phi+t)\\
            x_2 = r\sin(\phi+t)
        \end{cases}.
    \]
    Тогда на фазовом портрете траекториями будут концентрические окружности. Так как при движении точки по окружности радиус не зависит от $t$, то в качестве первого интеграла можно взять $u(x) = x_1^2 + x_2^2$.
    % Так как $f(x_0) = (0, 0)$, предположение теоремы нарушается.
    % Проверим, что следствие теоремы тоже нарушится: от противного, пусть существует первый интеграл $v(x_1, x_2)$ в окрестности $x_0$.
    % Мы ещё требуем невырожденность, поэтому
    % \[
    %     \left( \frac{\partial v}{\partial x_1}(0, 0), \frac{\partial v}{\partial x_2}(0, 0) \right) \ne (0, 0).
    % \]
    % Без ограничения общности $\frac{\partial v}{\partial x_1}(0, 0) > 0$, тогда это верно и в некоторой окрестности нуля.
    % По определению первого интеграла производная в силу системы должна быть тождественным нулём:
    % \[
    %     -x_2 \frac{\partial v}{\partial x_1}(x_1, x_2) + x_1 \frac{\partial v}{\partial x_2}(x_1, x_2) \equiv 0.
    % \]
    % Возьмём $x_1 = 0$ и $x_2 = \frac{1}{N}$, где $N$ --- какое-то достаточно большое число.
    % Тогда из доказанного выше $\frac{\partial v}{\partial x_1}(x_1, x_2) > 0$.
    % Вернёмся к тождеству выше:
    % \[
    %     -x_2 \frac{\partial v}{\partial x_1}(x_1, x_2) \equiv 0.
    % \]
    % Но мы взяли $x_1$, $x_2$ так, что оба множителя не равны нулю --- противоречие.

    % \QED
\end{itemize}

\textbf{Замечание.} Зачем нужны первые интегралы? Оказывается, с помощью них можно сводить автономные системы к алгебраическим уравнениям. Нестрого поясним, как это можно делать. Пусть $\Omega \subset \mathbb{R}^3$.
Возьмём два первых интеграла $u_1, u_2$, константы $c_1, c_2$ и рассмотрим следующую систему:
\[
    \begin{cases}
        u_1(x) = c_1\\
        u_2(x) = c_2
    \end{cases}.
\]
Оба уравнения задают поверхности, а решения системы составляют кривую, по которой пересекаются эти поверхности. Можно показать, что кусочки этой кривой будут являться траекториями автономной системы.

\textbf{Определение.} Пусть $D \subset \Omega$ открыто и функции $v_1, \dots, v_k \in C^1(D, \mathbb{R}), k < n$.
Тогда они называются \textit{функционально независимыми} на $D$, если ранг матрицы Якоби $\rank \frac{\partial(v_1, \dots, v_k)}{\partial(x_1, \dots, x_n)} \equiv k$.

\textbf{Замечание.} Из функциональной независимости следует линейная независимость, а вот обратное следствие неверно. Например, пусть $D = \mathbb{R}^2$. Возьмём $u(x_1, x_2) = x_1^2$. Заметим, что при $x_1 = 0$ ранг матрицы Якоби равен $0 < 1$, значит, нет функциональной независимости.

% \textbf{Теорема.} Для любой точки $(t_0, x_0) \in \Omega$ существует окрестность $D \subset \Omega$, а в ней --- независимые в $D$ первые интегралы $v_1, \dots, v_n$.

% \textbf{Доказательство.} Для любого $(t_0, \xi) \in D$ существует единственное непродолжаемое решение $\phi(\cdot, \xi)$ задачи Коши, ещё и непрерывно дифференцируемое:
% \[
%     \begin{cases}
%         x' = f(t, x) \\
%         x(t_0) = \xi 
%     \end{cases} .
% \]
% Решим уравнение $x - \phi(t, \xi) = 0$ относительно $\xi$ с параметрами $(t, x)$ в окрестности $(t_0, x_0)$, соответствующей решению $x = x_0$, $t = t_0$, $\xi = x_0$.
% Тогда $\phi(t_0, \xi) \equiv \xi$ по определению $\phi$.
% Теперь
% \[
%     \frac{\partial}{\partial \xi} (x - \phi(t, \xi)) \bigg|_{\substack{t = t_0 \\ x = x_0}}(x_0) = -E.
% \]
% ($x_0$ встречается дважды, так как сначала мы подставили параметр $x = x_0$, а потом неизвестную $\xi = x_0$)

% Следовательно, применима теорема о неявной функции: существует окрестность $D_1 \subset \Omega$ точки $(t_0, x_0)$ и отображение $V = (v_1, \dots, v_n): D_1 \to \mathbb R^n$, такое что:
% \begin{itemize}
%     \item $V \in C^1$.
%     \item Для всех $(t, x) \in D_1$ выполнено $x - \phi(t, V(t, x)) \equiv 0$.
%     \item $V(t_0, x_0) = x_0$.
%     \item Так как количество уравнений совпадает с количеством неизвестных, существует окрестность $\Delta$ точки $x_0$, такая что если $u \in \Delta$ и $x - \phi(t, u) = 0$, то $u = V(t, x)$.
% \end{itemize}
% Продифференцируем по $x$ второе свойство:
% \[
%     E \equiv \frac{\partial \phi}{\partial \xi}(t, V(t, x)) \cdot \frac{\partial V}{\partial x}(t, x).
% \]
% Подставим $(t_0, x_0)$: заметим, что $V(t_0, x_0) = x_0$, откуда это будет равно
% \[
%     E = \frac{\partial \phi}{\partial \xi}(t_0, x_0) \cdot \frac{\partial V}{\partial x}(t_0, x_0).
% \]
% Первый множитель равен $E$, поэтому
% \[
%     E = \frac{\partial V}{\partial x}(t_0, x_0).
% \]
% Отсюда ранг этой матрицы равен $n$, а значит, существует окрестность $D \subset D_1$, такая что в ней $\rank \left( \frac{\partial V}{\partial x} (t, x) \right) = n$.

% Пусть $x(\cdot)$ --- решение задачи Коши $x' = f(t, x)$, $x(t_0) = \xi$.
% По второму свойству $x(t) - \phi(t, V(t, x(t))) \equiv 0$.
% Более того, из обозначений $x(t) - \phi(t, \xi) \equiv 0$.
% Уменьшая область определения $x(\cdot)$, можно добиться того, чтобы $x(t)$ всегда попадал в $\Delta$, откуда по четвёртому свойству решение единственно и должно совпадать, поэтому $V(t, x(t)) \equiv \xi$, то есть $v_i(t, x(t)) \equiv \xi_i$ --- константы.

% \QED

% \subsection{Первые интегралы автономных систем}
% Пусть нам даны $n \in \mathbb N$, открытое $\Omega \subset \mathbb R^n$ и отображение $f: \Omega \to \mathbb R^n$, $f \in C^1$.
% Рассмотрим систему
% \begin{equation}
%     x' = f(x).
% \end{equation}

% \textbf{Теорема.} Для любого $x_0 \in \Omega$, такого что $f(x_0) \ne 0$ существует окрестность $D \subset \Omega$ точки $x_0$ и $n - 1$ независимых первых интегралов $v_i: D \to \mathbb R$.
% От предыдущего случая отличается тем, что $v_i$ не зависит от $t$.

% \textbf{Доказательство.}
% Так как $f(x_0) \ne 0$, у него существует ненулевая координата.
% Без ограничения общности это $n$-ая: $f_n(x_0) \ne 0$.
% Из непрерывности $f_n$ получаем, что $f_n(x) \ne 0$ в некоторой окрестности $x_0$.
% Рассмотрим неавтономную систему
% \begin{equation}
%     \frac{dx_i}{dx_n} = \frac{f_i(x)}{f_n(x)}.
% \end{equation}
% Здесь $(n - 1)$ уравнение, откуда по теореме из предыдущего пункта существует окрестность $D$ точки $x_0$ и независимые первые интегралы системы (3) $v_1, \dots, v_{n-1}: D \to \mathbb R$.

% Пусть $\phi(\cdot) = (\phi_1(\cdot), \dots, \phi_n(\cdot))$ --- какое-то решение системы (2), такое что для всех $t$ $\phi(t) \in D$, то есть $\phi_n'(t) = f_n(\phi(t)) \ne 0$
% Итак, мы получили строго монотонную функцию $\phi_n(t)$ на интервале --- по первому семестру матанализа существует обратная к ней функция $t(\phi_n)$.
% Обозначим $x_i(x_n) = \phi_i(t(x_n))$.
% Тогда
% \[
%     \frac{dx_i}{dx_n}(x_n) \equiv \frac{d \phi_i}{dt}(t(x_n)) \cdot \frac{dt}{dx_n}(x_n) \equiv f_i(\phi(t(x_n))) \cdot \frac{1}{\phi_n'(t(x_n))} \equiv
% \]
% \[
%     \equiv \frac{f_i(\phi(t(x_n)))}{f_n(\phi(t(x_n)))} \equiv \frac{f_i(x)}{f_n(x)}.
% \]
% Вернёмся к первым интегралам: по определению для всех $i$
% \[
%     v_i(\phi_1(t(x_n)), \dots, \phi_{n-1}(t(x_n)), x_n) \equiv const.
% \]
% Обозначая $\tau = t(x_n)$, получаем
% \[
%     v_i(\phi_1(\tau), \dots, \phi_n(\tau)) \equiv const.
% \]
% Значит, $v_i$ являются первыми интегралами системы (2).
% Проверим их независимость.
% Мы знаем, что они независимы в системе (3), тогда векторы
% \[
%     \left(\frac{\partial v_i}{\partial x_1}(x), \dots, \frac{\partial v_i}{\partial x_{n-1}}(x) \right)
% \]
% линейно независимы.
% Нам нужны $n$-мерные векторы, поэтому добавим к ним ещё одну координату:
% \[
%     \left(\frac{\partial v_i}{\partial x_1}(x), \dots, \frac{\partial v_i}{\partial x_{n-1}}(x), \frac{\partial v_i}{\partial x_n}(x) \right).
% \]
% При добавлении новой координаты линейная независимость не ломается, поэтому они подходят в систему (2).
% Таким образом, ранг матрицы 
% \[
%     \left( \frac{\partial v_i}{\partial x_j} \right)_{\substack{i = \overline {1, n - 1} \\ j = 1, n}}
% \]
% равен $n - 1$, то есть $v_1, \dots, v_{n-1}$ --- искомые первые интегралы.

% \QED

% \textbf{Замечание.} Зачем: возьмём $n = 2$ и первый интеграл $v_1$.
% Тогда кривая $v_1(x) = v_1(x_0)$ является фазовой траекторией.
% Аналогично в трёхмерном случае, но тогда будет пересечение поверхностей.

% \textbf{Пример.} Рассмотрим систему

\subsection{Множество всех первых интегралов автономной системы}
\textbf{Утверждение.} Если у нас есть $k$ первых интегралов $u_1(x), \dots, u_k(x)$ системы (1), то функция $F(u_1(x), \dots, u_k(x))$, где $F$ непрерывно дифференцируема, также является первым интегралом системы (1).

\textbf{Доказательство.} Действительно, пусть $x(t)$ --- это решение системы (1), тогда
\[
    F(u_1(x(t)), \dots, u_k(x(t))) \equiv F(\text{const}, \dots, \text{const}) \equiv \text{const}.
\]

\QED

Поскольку первых интегралов бесконечное количество (хотя бы потому, что все константы ими являются), то возникает вопрос: можно ли взять несколько первых интегралов и, используя утверждение выше, получить все возможные первые интегралы? Оказывается, что при определённых условиях так правда можно сделать.

\textbf{Теорема.} Пусть $x_0 \in \Omega$ и $f(x_0) \ne 0$. Тогда существуют окрестность $X(x_0)$ и $n-1$ функционально независимых первых интегралов $u_2, \dots, u_{n}\colon X(x_0) \to \mathbb{R}$ системы (1).

\textbf{Доказательство.} Докажем теорему для случая, когда у нас фазовые траектории являются прямыми, а потом сведём общий случай к этому с помощью теоремы о выпрямлении траекторий. Давайте всё формализуем.
\begin{enumerate}
    \item Так как $f(x_0) \ne 0$, то мы можем использовать теорему о выпрямлении траекторий. Значит, существует окрестности $X(x_0)$, $Y(0)$ и диффеоморфизм $\Psi\colon Y(0) \to X(x_0)$. Зададим для $i=\overline{2, n}$ функции $v_i\colon Y(0) \to \mathbb{R}$ по формуле $v_i(y) \coloneq y_i$.
    Тогда все $v_i$ --- это первые интегралы системы
    \begin{equation}
        \begin{cases}
            y_1' = 1\\
            y_2' = 0\\
            \vdots\\
            y_n' = 0
        \end{cases},
    \end{equation}
    так как любое решение этой системы имеет вид
    \[
        y(t) = \begin{pmatrix}
            t+C_1\\
            C_2\\
            \vdots\\
            C_n
        \end{pmatrix},
    \]
    а тогда $v_i(y(t)) \equiv c_i$. При этом все $v_i$ функционально независимы в окрестности $Y(0)$, так как $\grad v_i(y(t)) = (0, \dots, 1, \dots, 0)^T$, где $1$ стоит на месте $i$, тогда у матрицы Якоби как раз будет ранг $n-1$.
    \item Теперь введём функции $u_2, \dots, u_n\colon X(x_0) \to \mathbb{R}$ по формуле $u_i(x) \coloneq v_i(\Psi^{-1}(x))$. Тогда все $u_i$ --- это первые интегралы системы (1), так как для любого решения $x(t)$ системы (1) $\Psi^{-1}(x(t))$ будет решением системы (2), а тогда
    \[
        u_i(x(t)) \equiv v_i(\Psi^{-1}(x(t))) \equiv \text{const}.
    \]
    Ну и все $u_i$ функционально независимы на $X(x_0)$, так как $v_i$ функционально независимы, а $\Psi$ --- это диффеоморфизм $\Rightarrow$ его матрица Якоби невырожденная и при домножении на невырожденную матрицу ранг не меняется.

    \QED
\end{enumerate}


\textbf{Теорема.} Пусть $x_0 \in \Omega$ и $f(x_0) \ne 0$. Тогда для любых функционально независимых первых интегралов $u_1, \dots, u_{n-1}\colon X(x_0) \to \mathbb{R}$ системы (1) существуют окрестности $X'(x_0) \subset X(x_0)$ и $W((u_1(x_0), \dots, u_{n-1}(x_0)))$ такие, что
для любого первого интеграла $u\colon X'(x_0) \to \mathbb{R}$ системы (1) существует функция $F \in C^1(W, \mathbb{R})$: $u(x) \equiv F(u_1(x), \dots, u_{n-1}(x))$.

\textbf{Доказательство.} Здесь мы снова докажем теорему сначала для системы (2), а потом перенесём всё на систему (1) с помощью теоремы о выпрямлении траектории (обозначения из её формулировки снова в силе).
\begin{enumerate}
    \item Сначала поймём, что любой первый интеграл $v\colon Y(0) \to \mathbb{R}$ системы (2) не зависят от $y_1$, так как для любого решения $y(t)$ получаем, что $v(t, y_2, \dots, y_n) \equiv \text{const}$ для любого $t \in (-\varepsilon, \varepsilon)$, при этом $y_2, \dots, y_n \in O^{n-1}(0, \varepsilon)$.

    Пусть $v_1, \dots, v_{n-1}\colon Y(0) \to \mathbb{R}$ --- функционально независимые первые интегралы системы (2). Определим отображение $\Phi\colon O^{n-1}(0, \varepsilon) \to \mathbb{R}^{n-1}$ следующим образом:
    \[
        \Phi(y_2, \dots, y_n) \coloneq \begin{pmatrix}
            v_1(y_1, \dots, y_n)\\
            \vdots\\
            v_{n-1}(y_1, \dots, y_n)
        \end{pmatrix}.
    \]
    Тогда в силу независимости $v_i$ на $Y(0)$ матрица Якоби отображения $\Phi$ имеет максимальный ранг $n-1$ $\Rightarrow$ по теореме об обратной функции существуют окрестности $V(0) \subset O^{n-1}(0, \varepsilon)$ и $W((v_1(0), \dots, v_{n-1}(0)))$ такие, что $\Phi$ --- диффеоморфизм между $V$ и $W$.

    Пусть $v\colon (-\varepsilon, \varepsilon) \times V \to \mathbb{R}$ --- первый интеграл системы (2). Тогда определим отображение 
    \[
        g(y) \coloneq v(y_1, \Phi^{-1}(\Phi(y_2, \dots, y_n))) \equiv v(y_1, \Phi^{-1}(v_1(y), \dots, v_{n-1}(y))).
    \]
    Теперь, если $z_2, \dots, z_n \in W$, то можем построить искомую функцию $F$:
    \[
        F(z_2, \dots, z_n) \coloneq v(y_1, \Phi^{-1}(z_2, \dots, z_n)).
    \]
    Тогда как раз получаем, что $v(y) \equiv F(v_1(y), \dots, v_{n-1}(y))$, значит, для системы (2) мы теорему доказали.
    \item Докажем для общего случая. Пусть у нас есть функционально независимые первые интегралы $u_1, \dots, u_{n-1}\colon X(x_0) \to \mathbb{R}$. Тогда функции $v_1, \dots, v_{n-1}$, определённые как $v_i(y) \coloneq u_i(\Psi(y))$, где $y \in (-\varepsilon, \varepsilon) \times V$, --- это функционально независимые первые интегралы системы (2).
    Положим $X'(x_0) \coloneq \Psi((-\varepsilon, \varepsilon) \times V)$. Возьмём какой-нибудь первый интеграл $u\colon X'(x_0) \to \mathbb{R}$ и определим первый интеграл $v(y) \coloneq u(\Psi(y))$. Тогда из первого пункта доказательства существуют окрестность $W((v_1(0), \dots, v_{n-1}(0)))$ и $F \in C^1(W, \mathbb{R})$: $v(y) \equiv F(v_1(y), \dots, v_{n-1}(y))$. Перепишем согласно определению $v$ и $v_i$:
    \[
        u(\Psi(y)) \equiv F(u_1(\Psi(y)), \dots, u_{n-1}(\Psi(y))).
    \]
    Тогда $u(x) \equiv F(u_1(x), \dots, u_{n-1}(x))$, где $x = \Psi(y), y \in (-\varepsilon, \varepsilon) \times V$, что и требовалось.

    \QED
\end{enumerate}

% \textbf{Теорема.} Пусть $D$ --- окрестность точки $(t_0, x_0)$, $v_1, \dots, v_n: D \to \mathbb R$ --- независимые первые интегралы системы (1).
% Обозначим $v := (v_1, \dots, v_n): D \to \mathbb R^n$ и $c_0 = v(t_0, x_0)$.
% Тогда
% \begin{itemize}
%     \item Если $x(\cdot)$ --- решение задачи Коши $x' = f(t, x)$, $x(t_0) = x_0$, то $x(\cdot)$ является решением алгебраической системы $v(t_0, x) = c_0$.
%     \item Если $\phi(\cdot, c)$ --- это решение алгебраической системы $v(t, x) = c$ и $c$ достаточно близко к $c_0$, то $\phi(\cdot, c)$ --- это решение системы (1).
% \end{itemize}
% В каком-то смысле дифференциальная система и алгебраическая система на первых интегралах эквивалентны.

% \textbf{Доказательство.} Первая часть: для любого $t$
% \[
%     v(t, x(t)) \equiv (v_i(t, x(t)))_{i=\overline{1, n}} \equiv const = v(t_0, x(t_0)) = c_0,
% \]
% что и требовалось.

% Вторая часть: пусть $v_1, \dots, v_n$ --- независимые первые интегралы, тогда ранг матрицы
% \[
%     \left( \frac{\partial v_i}{\partial x_j}(t_0, x_0) \right)_{i,j = \overline{1, n}}
% \]
% равен $n$.
% Пусть $x(\cdot)$ --- решение системы (1).
% По первому пункту $x$ является решением системы $v(t, x) = v(t_0, x_0)$.
% Тогда по теореме о неявной функции $\phi(\cdot, c)$, как второе решение этой системы, совпадает с $x(t)$.
% Следовательно, $\phi(t, v(t_0, x(t_0)))$ --- решение системы (1).

% (Что здесь происходит...)

% \QED


% \textbf{Утверждение.} Пусть $D$ --- окрестность $(t_0, x_0)$, $v_1, \dots, v_n: D \to \mathbb R$ --- независимые первые интегралы системы (1), $v$ и $c_0$ из теоремы.
% Тогда существует окрестность $D' \subset D$ точки $(t_0, x_0)$, такая что любой первый интеграл $\omega: D' \to \mathbb R$ системы (1) представим в виде $\omega(t, x) = F(v(t, x))$.

% \textbf{Доказательство.} Пусть $\phi(t, c)$ --- решение системы $v(t, x) = c$, тогда $\phi(\cdot, c)$ по теореме является решением системы (1).
% Пусть $\phi(t, v(t, \xi))$ --- решение системы $v(t, x) = v(t, \xi)$.
% Тогда по теореме о неявной функции существует окрестность $D' \subset D$, такая что $x = \xi$ --- единственное решение для всех $x$, достаточно близких к $x_0$, такое что для всех $(t, \xi) \in D'$ выполняется $\phi(t, v(t, \xi)) \equiv \xi$.

% Положим $F(c) := \omega(t_0, \phi(t_0, c))$.
% По определению первого интеграла $\omega(t, \phi(t, c))$ --- константа по $t$.
% Подставим $t_0$: $\omega(t_0, \phi(t_0, c)) = F(c)$.
% Теперь подставим $c = v(t, \xi)$: $\omega(t, \xi) \equiv F(v(t, \xi))$ --- ровно искомое тождество.

% \QED

% Смысл доказательств --- переход от дифференциальных уравнений к алгебраическим и последующее применение теоремы о неявной функции.

% \subsection{Множество первых интегралов автономных систем}
% Будем доказывать те же теоремы для автономных систем.
% Преамбула такая же, как и в пункте 2.

% \textbf{Лемма.} Пусть $f_n(x_0) \ne 0$, где $x_0 \in \Omega$, функции $v_1, \dots, v_{n-1}: \Omega \to \mathbb R$ --- независимые первые интегралы системы (2).
% Тогда $v_1, \dots, v_{n-1}$ --- независимые первые интегралы системы $\frac{d x_i}{d x_n} = \frac{f_i(x)}{f_n(x)}$.
% Это аналог теоремы из пункта 2, но теперь в роли времени выступает $x_n$.

% \textbf{Доказательство.} Из определения первого интеграла для любого $j$ выполняется
% \[
%     \sum_{i=1}^{n} \frac{\partial v_j}{\partial x_i}(x) f_i(x) \equiv 0.
% \]
% Тогда можно разделить на $f_n(x)$:
% \[
%     \sum_{i=1}^{n-1} \frac{\partial v_j}{\partial x_i}(x) \frac{f_i(x)}{f_n(x)} + \frac{\partial v_j}{\partial x_n}(x) \equiv 0.
% \]
% По признаку для неавтономных систем получаем, что $v_j$ являются первыми интегралами системы $\frac{dx_i}{dx_n}$. Докажем их независимость.
% Рассмотрим матрицу
% \[
%     \left( \frac{\partial v_j}{\partial x_i}(x) \right)_{\substack{i = \overline{1, n} \\ j = \overline{1, n - 1}}}.
% \]
% Её ранг равен $n - 1$ для всех $x$, причём её последняя строка, из доказанного, выражается через первые $n - 1$.
% Следовательно, ранг матрицы
% \[
%     \left( \frac{\partial v_j}{\partial x_i}(x) \right)_{\substack{i = \overline{1, n - 1} \\ j = \overline{1, n - 1}}}
% \]
% равен $n - 1$, что доказывает независимость первых интегралов.

% \QED

% \textbf{Теорема.} (О множестве первых интегралов) Пусть $v_1, \dots, v_{n-1}: D \to \mathbb R$ --- независимые первые интегралы автономной системы (2), $x_0 \in D$, $f(x_0) \ne 0$.
% Тогда существует окрестность $D' \subset D$ точки $x_0$, такая что для любого первого интеграла системы (2) $\omega: D' \to \mathbb R$ существует функция $F \in C^1$, такая что $\omega(x) \equiv F(v_1(x), \dots, v_{n-1}(x))$.

% \textbf{Доказательство.} Без ограничения общности $f_n(x_0) \ne 0$, причём, уменьшая, при необходимости, $D$, это верно на всём $D$.
% По лемме $v_1, \dots, v_{n-1}$ является первыми интегралами системы $\frac{dx_i}{dx_n}$, откуда из утверждения для неавтономных систем существует окрестность $D'$ точки $x_0$, такая что выполняется всё, что нужно.

% \QED