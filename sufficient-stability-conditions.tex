\setcounter{equation}{0}
\section{Устойчивость по Ляпунову и асимптотическая устойчивость}
\subsection{Определение и примеры}
Снова заданы открытое множество $\Omega \subset \mathbb R^n$, отображение $f \in C^1(\Omega, \mathbb R^n)$, положение равновесия $\widehat{x} \in \Omega$ и автономная система
\begin{equation}
    x' = f(x).
\end{equation}
Пусть $\phi(\cdot, \xi)$ --- непродолжаемое решение задачи Коши 
\[
\begin{cases}
    x' = f(x)\\
    x(0) = \xi
\end{cases}.
\]
\textbf{Определение.} Положение равновесия $\widehat{x}$ называется \textit{устойчивым по Ляпунову}, если:
\begin{enumerate}
    \item cуществует $r > 0$ такое, что для любого $\xi \in O(\widehat{x}, r)$ отображение $\phi(\cdot, \xi)$ определено на $[0, +\infty)$;
    \item для любого $\varepsilon > 0$ существует $\delta > 0$ такое, что для всех $\xi \in O(\widehat{x}, \delta)$ и для всех $t \in [0, +\infty)$ верно $\phi(t, \xi) \in O(\widehat{x}, \varepsilon)$.
\end{enumerate}

\textbf{Определение.} Положение равновесия $\widehat{x}$ называется \textit{асимптотически устойчивым}, если:
\begin{enumerate}
    \item оно устойчиво по Ляпунову;
    \item cуществует $d > 0$ такое, что для всех $\xi \in O(\widehat{x}, d)$ функция $\phi(t, \xi) \to \widehat{x}$ при $t \to +\infty$.
\end{enumerate}

\textbf{Примеры.} 
\begin{itemize}
    \item Пусть $\Omega \subset \mathbb R$ и $\widehat{x}$ --- изолированное положение равновесия, то есть существует окрестность $\widehat{x}$ такая, что в ней нет других положений равновесия.
        Тогда один из возможных случаев: это когда в этой окрестности функция $f \geq 0$ и равна нулю только в точке $\widehat{x}$.
        
        Посмотрим на интегральные кривые. Есть горизонтальная прямая, соответствующая решению
        $x(t) \equiv \widehat{x}$. Если берём начальное условие $\xi < \widehat{x}$, то соответствующее решение будет монотонно возрастать в силу положительности производной, тогда из теоремы о существовании и единственности следует, что горизонтальная прямая будет его асимптотой. Если же берём начальное условие выше $\widehat{x}$, то решение снова будет возрастать. Тогда здесь нет даже устойчивости по Ляпунову.
        А вот если рассмотреть случай, когда функция $f(x) > 0$ при $x < \widehat{x}$ и $f(x) < 0$ при $x > \widehat{x}$, то аналогичным образом можно показать, что там будет асимптотическая устойчивость, а тогда и устойчивость по Ляпунову.
    \item Пусть теперь $\Omega \subset \mathbb{R}^2$, $f(x) = Ax$ и $\widehat{x} = 0$. Возвращаясь к случаям из предыдущего параграфа, устойчивость по Ляпунову будет на всех устойчивых портретах, а ещё для центра. Они же, но уже за исключением центра, будут и асимптотически устойчивы.
    
\end{itemize}
\subsection{Достаточные условия устойчивости}
\subsubsection{Устойчивость линейных систем}
Пусть даны матрица $A \in \mathbb R^{n \times n}$ и система
\begin{equation}
    x' = Ax.
\end{equation}
% Пусть $X(t)$ --- фундаментальная система решений системы (2), такая что $X(0) = I$ (единичная).
% Она существует, так как можно рассмотреть $n$ задач Коши $x' = Ax$ и $x(0) = e_i$, где $e_i$ --- $i$-ый базисный вектор $\mathbb R^n$.
Пусть в ЖНФ матрицы $A$ есть жордановы клетки $K_1, \dots, K_m$, причём для каждой клетки $K_j$ её размер равен $k_j$ и ей соответствует собственное число $\lambda_j = \alpha_j + i\beta_j$. Без ограничения общности будем считать, что $\lambda_1, \dots, \lambda_s \in \mathbb C$, при этом им соответствуют сопряжённые числа $\lambda_{s+1} = \overline{\lambda_1}, \dots, \lambda_{2s} = \overline{\lambda_s}$, а $\lambda_{2s+1}, \dots, \lambda_m \in \mathbb R$.

\textbf{Теорема.} 
\begin{enumerate}
    \item Если все $\re(\lambda_j) < 0$, то $\widehat{x} = 0$ --- асимптотически устойчивое положение равновесия.
    \item Если все $\re(\lambda_j) \le 0$, а для $j$ таких, что $\re(\lambda_j) = 0$, выполнено $k_j = 1$, то $\widehat{x} = 0$ устойчиво по Ляпунову, но не асимптотически устойчиво.
    \item В остальных случаях $\widehat{x} = 0$ не устойчиво по Ляпунову.
\end{enumerate}

\textbf{Доказательство.} Из прошлого семестра мы знаем, что любое решение $x(t)$ системы (2) представимо в виде
\[
    x(t) = \sum_{j=1}^{s} P_j(t) e^{\alpha_j t}\cos(\beta_j t) + \sum_{j=s+1}^{2s} P_j(t) e^{\alpha_{j-s} t}\sin(\beta_{j-s} t) + \sum_{j=2s+1}^{m} P_j(t) e^{\lambda_j t},
\]
причём $\deg P_j \le k_j - 1$.

\begin{enumerate}
\item Из условия $\re \lambda_j < 0$ следует, что $|x(t)| \to 0$ при $t \to +\infty$. Пусть $X(t)$ --- ФМР. Так как столбцы $X(t)$ являются решениями, $\|X(t)\| \to 0$ при $t \to +\infty$.
Тогда $\|X(t)\|$ равномерно ограничена некоторым числом $c > 0$. Кроме того, имеем следующее неравенство:
\[
    |\phi(t, \xi)| = |X(t) \xi| \le \| X(t) \| \cdot |\xi|.
\]
Заметим, что $\phi(\cdot, \xi)$ при любом $\xi$ определено на $[0, +\infty)$, так как (2) является линейной системой с постоянными коэффициентами. Зафиксируем $\varepsilon > 0$. Чтобы выполнялся второй пункт из определения устойчивости по Ляпунову, можно взять $\delta = \frac{\varepsilon}{2c}$, тогда
при $\xi \in O(0, \delta)$ из неравенства выше получаем, что 
\[
    |\phi(t, \xi)| \leq c \cdot \frac{\varepsilon}{2c} = \frac{\varepsilon}{2} < \varepsilon.
\]
Асимптотическая устойчивость следует из того, что $\|X(t)\| \to 0$, а $|\xi|$ ограничен.

\item Если $\re(\lambda_j) < 0$, то соответствующее слагаемое стремится к нулю $\Rightarrow$ ограничено на $[0, +\infty)$.
Остаётся случай $\re(\lambda_j) = 0 \Rightarrow k_j = 1$. Тут мы пользуемся тем, что $\deg P_j(t) \le 1 - 1 = 0$ $\Rightarrow$ многочлен $P_j$ --- это просто константа. Тогда и всё соответствующее слагаемое будет ограничено.
Значит, каждое решение $x(t)$ на $[0, +\infty)$ ограничено, тогда существует такое число $c > 0$, что $\|X(t)\| \le c$ для всех $t \in [0, +\infty)$. Далее работает такое же рассуждение, как в первой части, поэтому получаем устойчивость по Ляпунову. Вывод об асимптотической устойчивости мы так сразу сделать не можем, так как не все решения стремятся к нулю. Покажем, что здесь её просто не может быть, предъявив явное решение.

Пусть $j$ таково, что $\re \lambda_j = 0, k_j = 1$. Тогда если $\lambda_j \in \mathbb C$, то берём решение $x(t) \coloneq r(v_j \cos(\beta_j t) + u_j\sin(\beta_j t))$, где $r > 0$, $u_j, v_j \in \mathbb R^n$. Уменьшая $r$, мы можем попасть в сколь угодно малую окрестность нуля, но при этом $x(t) \nrightarrow 0$. Если же $\lambda_j \in \mathbb R$, то возьмём решение $x(t) \coloneq rv_j$. Оно опять же не стремится к нулю.

\item Пусть существует $j$ такое, что $Re(\lambda_j) > 0$. Если $\lambda_j \in \mathbb C$, то берём решение $x(t) \coloneq re^{\alpha_j t} (P_j(t)\cos(\beta_j t) + P_{j+s}(t)\sin(\beta_j t))$. Для любого $r > 0$ оно не ограничено $\Rightarrow$ нет устойчивости по Ляпунову.
Если же $\lambda_j \in \mathbb R$, то подойдёт решение $x(t) \coloneq r e^{\lambda_j t}v_j$, $r > 0$, $v_j \in \mathbb R^n$.

Остался случай, когда все $\re \lambda_j \le 0$ и существует $j$ такое, что $\re \lambda_j = 0$, но при этом $k_j \ge 2$. Тогда если $\lambda_j \in \mathbb C$, то есть неограниченное комплексное решение $x(t) \coloneq r(a+bt)e^{\lambda_jt}$, $r > 0$, $b \ne 0$. Тогда либо $\re x(t)$, либо $\im x(t)$ не ограничено $\Rightarrow$ нет устойчивости по Ляпунову.
Если же $\lambda_j \in \mathbb R$, то возьмём решение $x(t) \coloneq r(at + b)$, $a, b \in \mathbb R^n$, $a \ne 0$, $r > 0$. Снова $x(t)$ не ограничено $\Rightarrow$ нулевое положение равновесия не устойчиво по Ляпунову.
\end{enumerate}

\QED

% \subsubsection{Устойчивость нелинейных систем}
% Пусть $\Omega \subset \mathbb{R}^n$ открыто, $f \in C^1(\Omega, \mathbb{R}^n)$, $\widehat{x} \in \Omega$ --- положение равновесия.
% Рассмотрим задачу Коши
% \[
% \begin{cases}
%     x' = f(x)\\
%     x(0) = x_0
% \end{cases}.
% \]
% Пусть $\phi(\cdot, x_0)$ --- её непродолжаемое решение. Обозначим матрицу Якоби $A \coloneq \frac{\partial f}{\partial x}(\widehat{x})$. Пусть $\lambda_1, \dots, \lambda_k$ --- собственные числа матрицы $A$.

% \textbf{Теорема.} (Ляпунова) Если все $\re \lambda_j < 0$, то положение равновесия $\widehat{x}$ асимптотически устойчиво.

% \textbf{Замечания.}
% \begin{enumerate}
%   \item Помним, что можно разложить функцию по формуле Тейлора:
%   \[
%   f(x) \equiv A(x- \widehat{x}) + r(x-\widehat{x}),
%   \]
%   при этом $r(x-\widehat{x}) = o(|x-\widehat{x}|)$ при $|x-\widehat{x}| \to 0$.
% \item Вспомним лемму о дифференциальном неравенстве. Она гласит, что если нам дан интервал $I \subset \mathbb{R}$, $t_0 \in I$, $z \in C^1(I, \mathbb{R}^n)$, $A > 0$, $B \ge 0$,
% а также на всём $I$ выполнено неравенство $|z'(t)| \le A|z(t)| + B$, то верно следующее неравенство:
% \[
%   |z(t)| \le |z(t_0)|e^{A(t-t_0)} + \frac{B}{A}(e^{A(t-t_0)} - 1).
% \]
% Нам оно нужно в частном случае, когда $t_0 = 0$. Сделаем следующее: подставим это неравенство в неравенство из условия леммы, получим следующее:
% \[
%   |z'(t)| \le A|z(0)|e^{At} + Be^{At}.
% \]

% \end{enumerate}

% Рассмотрим систему
% \begin{equation}
%     x' = f(x).
% \end{equation}

% \textbf{Определение.} Пусть $\Pi \subset \Omega$ открыто, а функция $v \in C^1(\Pi, \mathbb{R})$. Тогда \textit{производной в силу системы (1)} называется функция
% \[
%   \left.\frac{dv}{dt}\right|_{(1)}(x) \coloneq \left< \frac{\partial v}{\partial x}(x), f(x) \right> = \sum_{j=1}^n{\frac{\partial v}{\partial x_j}(x)f_j(x)}.
% \]
% Пусть $x(t)$ --- решение системы (1). Тогда
% \[
%     \frac{dv}{dt}(x(t)) \equiv \left< \frac{\partial v}{\partial x}(x(t)), x'(t) \right> \equiv \left< \frac{\partial v}{\partial x}(x(t)), f(x(t)) \right> \equiv \left.\frac{dv}{dt}\right|_{(1)}(x(t)).
% \]

% \textbf{Определение.} Функция $v$ в следующих двух теоремах называется \textit{функцией Ляпунова}.

% \textbf{Теорема.} (Ляпунова об устойчивости) Пусть существуют $\varepsilon > 0$ и функция $v \in C^1(O(\widehat{x}, \varepsilon), \mathbb R)$ такие, что:
% \begin{enumerate}
%   \item $v(\widehat{x}) = 0$
%   \item $v(x) > 0$ при $x \ne \widehat{x}$
%   \item для всех $x$ выполнено $\left.\frac{dv}{dt}\right|_{(1)}(x) \le 0$.
% \end{enumerate}
% Тогда $\widehat{x}$ устойчиво по Ляпунову.

% \textbf{Теорема.} (Ляпунова об асимптотической устойчивости) Пусть существуют $\varepsilon > 0$ и функция $v \in C^1(O(\widehat{x}, \varepsilon), \mathbb R)$ такие, что:
% \begin{enumerate}
%   \item выполнены условия 1 и 2 из предыдущей теоремы
%   \item для всех $x \ne \widehat{x}$ выполнено $\left.\frac{dv}{dt}\right|_{(1)}(x) < 0$.
% \end{enumerate}
% Тогда $\widehat{x}$ асимптотически устойчиво.

% \textbf{Примеры.}
% \begin{itemize}
%   \item Рассмотрим уравнение $x' = -x^3$, $\widehat{x} = 0$. Тогда в качестве функции Ляпунова можно взять $v(x) = x^2$. Тогда для $x \ne 0$ верно следующее:
%   \[
%     \left.\frac{dv}{dv}\right|_{(1)}(x) = 2x \cdot (-x^3) = -2x^4 < 0.
%   \]
%   Получаем, что выполнены все три условия теоремы Ляпунова об асимптотической устойчивости $\Rightarrow$ $\widehat{x} = 0$ асимптотически устойчиво.
%   \item Рассмотрим систему 
%   \[
%       \begin{cases}
%           x_1' = -x_2 \\
%           x_2' = x_1
%       \end{cases}
%   \]
%   и $\widehat{x} = (0, 0)^T$.
%   Положим $v(x) = x_1^2 + x_2^2$. Тогда $v(0) = 0$ и $v(x) > 0$ при $x \ne 0$.
%   Найдём производную:
%   \[
%       \left.\frac{dv}{dt}\right|_{(1)} (x) = \left< (2x_1, 2x_2)^T, (-x_2, x_1)^T \right> \equiv 0.
%   \]
%   Следовательно, по теореме Ляпунова об устойчивости $\widehat{x}$ устойчиво.
%   Но асимптотической устойчивости нет, так как фазовым портретом этой системы является центр.
% \end{itemize}